\documentclass[a4paper]{article}
\usepackage[top=3.5cm,right=2.75cm,left=2.75cm,bottom=3.5cm]{geometry}

\usepackage[german]{babel}
%\usepackage{umlaut}
\usepackage{graphicx}
\usepackage{amsfonts}
%\usepackage{fullpage}
\usepackage[latin1]{inputenc}

\def\header#1#2#3#4{\pagestyle{empty}
\noindent
\begin{minipage}[t]{0.6\textwidth}
\begin{flushleft}
\bf \"Ubungen zur Datenkompression\\
WSI f\"ur Informatik\\
Krebs/Deininger
\end{flushleft}
\end{minipage}
\begin{minipage}[t]{0.4\textwidth}
\begin{flushright}
\bf Sommersemester 2011\\
Universit\"at T\"ubingen\\
#2 %Datum eintragen
\end{flushright}
\end{minipage}

\begin{center}
{\Large\bf Blatt #1}

{(Abgabe am #3)}
\end{center}
}

\begin{document}
\header{4}{19.05.2011}{26.5.2011}{}

\bigskip

{\bf Zum Blatt}\\
Auf diesen Blatt sollen Sie sich zwei der Aufgaben aussuchen und diese
bearbeiten. Bei Abgabe von allen drei Aufgaben z"ahlen die zwei besten.

\bigskip

{\bf Aufgabe 1  \quad(Burrows-Wheeler) }\\
\begin{enumerate}
\item Schreiben Sie Routinen zur Berechnung der {\em Burrows-Wheeler}-Transformation und ihrer Umkehrung.
\item Warum ist es sinnvoll, nach der {\em Burrows-Wheeler}-Transformation die {\em Move to Front}-Transformation anzuwenden?
\item Schreiben Sie ein Programm, dass zun"achst die  {\em Burrows-Wheeler}-Transformation und dann die {\em Move to Front}-Transformation anwendet.
\item Testen Sie Ihr Programm an den Eingabedaten von Blatt 1.
\item Lassen sich durch die Anwendung beider Verfahren Vorteile bei der Datenkompression erwarten?
\end{enumerate}

\bigskip

{\bf Aufgabe 2  \quad(LZ77)}\\
\begin{enumerate}
\item Warum ist LZ77 (nicht) zur online Komprimierung geeignet? Also z.B. zur Komprimierung von interaktiven Netzwerkverbindungen wie SSH.
\item\label{a} Implementieren Sie den {\bf LZ77}-Algorithmus. Beim Erstellen der Klasse sollen die Gr"o{\ss}en des Absuch- und des Kodierpuffers w"ahlbar sein.

Die Ausgabe des Algorithmus sind Tripel wie auf den Folien beschrieben. Das heisst es ist keine Komprimierung oder Bin"arkodierung dieser Tripel notwendig.

\item Implementieren sie die Dekodierung zu Teilaufgabe \ref{a}.
\item Experimentieren Sie mit Ihrem Algorithmus mit verschiedenen Eingaben und verschiedenen Gr"o{\ss}en f"ur die Gr"o{\ss}en von Absuch- und Kodierpuffer. "Uberlegen Sie sich sinnvolle Gr"o{\ss}en, zu Testzwecken (kurze Begr"undung) und analysieren Sie die Ausgaben im Hinblick auf zu erreichende Kompressionsraten.

"Uberlegen Sie, wie sie die Ausgabe ihres Algorithmus kodieren k"onnen, um (bei geeigneten Eingaben) eine tats"achliche Kompression zu erreichen.

\item(*) Implementieren Sie ihren Algorithmus so, dass Sie tats"achlich Daten komprimieren k"onnen.
\end{enumerate}

\bigskip

{\bf Aufgabe 3  \quad(LZW)}\\
\begin{enumerate}
\item Warum ist LZW (nicht) zur online Komprimierung geeignet? Also z.B. zur Komprimierung von interaktiven Netzwerkverbindungen wie SSH.
\item\label{3a} Implementieren Sie die den {\bf LZW}-Algorithmus. Beim Erstellen der Klasse soll als Parameter die Gr"o{\ss}e des W"orterbuchs (=Anzahl der Eintr"age) "ubergeben werden k"onnen.

Die Ausgabe des Algorithmus sind Indizes wie auf den Folien beschrieben. Das heisst es ist keine Komprimierung oder Bin"arkodierung dieser Indizes notwendig.

\item Implementieren sie die Dekodierung zu Teilaufgabe \ref{3a}.
\item Experimentieren Sie mit Ihrem Algorithmus mit verschiedenen Eingaben und verschiedenen Gr"o{\ss}en f"ur das W"orterbuch. "Uberlegen Sie sich sinnvolle Gr"o{\ss}en, zu Testzwecken (kurze Begr"undung) und analysieren Sie die Ausgaben im Hinblick auf zu erreichende Kompressionsraten.

"Uberlegen Sie, wie sie die Ausgabe ihres Algorithmus kodieren k"onnen, um (bei geeigneten Eingaben) eine tats"achliche Kompression zu erreichen.

\item(*) Implementieren Sie ihren Algorithmus so, dass Sie tats"achlich Daten komprimieren k"onnen.
\end{enumerate}

\end{document}

