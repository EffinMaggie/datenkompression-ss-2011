\documentclass[a4paper]{article}
\usepackage[top=3.5cm,right=2.75cm,left=2.75cm,bottom=3.5cm]{geometry}

\usepackage[german]{babel}
%\usepackage{umlaut}
\usepackage{graphicx}
\usepackage{amsfonts}
%\usepackage{fullpage}
\usepackage[latin1]{inputenc}

\def\header#1#2#3#4{\pagestyle{empty}
\noindent
\begin{minipage}[t]{0.6\textwidth}
\begin{flushleft}
\bf \"Ubungen zur Datenkompression\\
WSI f\"ur Informatik\\
Krebs/Deininger
\end{flushleft}
\end{minipage}
\begin{minipage}[t]{0.4\textwidth}
\begin{flushright}
\bf Sommersemester 2011\\
Universit\"at T\"ubingen\\
#2 %Datum eintragen
\end{flushright}
\end{minipage}

\begin{center}
{\Large\bf Blatt #1}

{(Abgabe am #3)}
\end{center}
}

\begin{document}
\header{6}{17.06.2011}{03.07.2011}{}

\bigskip

{\bf Zum Blatt}\\
Ziel dieses Blattes ist die Komprimierung von Bildern mit einer 8-bit Palette.

\bigskip

{\bf Aufgabe 1  \quad(Bildkomprimierung)}\\
Laden Sie von der "Ubungsseite das Material f"ur dieses Blatt herunter. Es
enth"alt zwei Paare an Bildern, jeweils im PNG- und im GIF-Format. Die Bilder
sind im Original einmal 493x329 Pixel und einmal 1024x768 Pixel. Daraus ergeben
sich unkomprimierte Gr"ossen von 162197 Byte und 786432 Byte (zzgl. Daten f"ur
die Palette).

Wenn Sie sich die komprimierten Versionen ansehen, erkennen Sie, dass die Bilder
einmal 13kB (GIF) bzw. 5kB (PNG) und einmal 455kB (GIF) bzw. 384kB (PNG). Beide
Formate sind verlustfrei.

Ihre Aufgabe:
\begin{enumerate}
\item Schreiben Sie ein Programm, welches ein Bild aus einer Datei einliest, und
      dieses (auf beliebige Weise) komprimiert und in einer anderen Datei
      ablegt. Die Zieldatei soll f"ur die Beispielbilder kleiner sein als eine
      unkomprimierte Version.
\item Schreiben Sie ein weiteres Programm, welches die Umkehrung zum vorherigen
      Aufgabenschritt berechnet: Eingabe ist ein Bild in dem von Ihnen erzeugten
      Format, Ausgabe ist in einem ''normalen'' Bildformat.
\end{enumerate}

Beachten Sie:
\begin{enumerate}
\item Sie m"ussen sich die Palette merken (in der Datei ablegen) um das Bild
      wieder auslesen zu k"onnen. Gehen Sie hierbei von drei 8-bit Farbkan"alen
      aus, also von 3 Byte pro Farbe. Welche drei Farben Sie w"ahlen bleibt
      prinzipiell Ihnen "uberlassen, das resultierende Gamut sollte aber
      mindestens den RGB Farbraum einschliessen.
\item Ihre Programme m"ussen auf beliebige 8-bit Palettenbilder anwendbar sein.
\item Verwenden Sie zum Einlesen des Originalbildes sowie zum Schreiben des
      dekodierten Bildes jeweils Sprachkonstrukte Ihrer ausgew"ahlten
      Programmiersprache. Ihre Sprache wird Ihnen dabei normalerweise auch
      Objekte oder Datentypen zur Manipulation des Bildpuffers vorgeben.
\item Die Ausgabe Ihres Kompressionsprogramms m"ussen Sie ''von Hand'' ablegen
      und sp"ater wieder auslesen. Sie d"urfen also nicht einfach den
      PNG-Kodierer ihrer Sprache verwenden; Serialisierungsfunktionen sind
      erlaubt, werden Ihnen aber vermutlich nur bedingt weiterhelfen.
\end{enumerate}

\bigskip

{\bf Aufgabe 2  \quad(Bildkomprimierung)}\\
Betrachten Sie Ihre Programme aus Aufgabe 1. Ihre Aufgabe ist es, diese
Programme so zu erweitern, dass sie bei den Testbildern besser komprimieren
(kleinere Ergebnisse liefern) als GIF.

F"ur Anregungen k"onnen Sie sich die Spezifikation von PNG ansehen, z.B. die
verschiedenen Filtertypen die im Standard vorgesehen sind. Die Spezifikation
f"ur PNG ist unter ''http://www.w3.org/TR/PNG/'' erreichbar, Kapitel 9 enth"alt
Details zu den Filtern, Kapitel 10 enth"alt einen kurzen Abriss der verwendeten
Kompressionsmethode.

\end{document}

