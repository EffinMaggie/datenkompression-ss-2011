\documentclass[a4paper]{article}
\usepackage[top=3.5cm,right=2.75cm,left=2.75cm,bottom=3.5cm]{geometry}

\usepackage[german]{babel}
%\usepackage{umlaut}
\usepackage{graphicx}
\usepackage{amsfonts}
\usepackage[latin1]{inputenc}

\def\header#1#2#3#4{\pagestyle{empty}
\noindent
\begin{minipage}[t]{0.6\textwidth}
\begin{flushleft}
\bf \"Ubungen zur Datenkompression\\
WSI f\"ur Informatik\\
Krebs/Deininger
\end{flushleft}
\end{minipage}
\begin{minipage}[t]{0.4\textwidth}
\begin{flushright}
\bf Sommersemester 2011\\
Universit\"at T\"ubingen\\
#2 %Datum eintragen
\end{flushright}
\end{minipage}

\begin{center}
{\Large\bf Blatt #1}

{(Abgabe am #3)}
\end{center}
}

\begin{document}
\header{1}{14.4.2011}{5.5.2011}{}

Beachten Sie bitte die Hinweise zu den "Ubungen!

\bigskip

{\bf Aufgabe 1  \quad(Entropie)}\\
Eine Informationsquelle sende Buchstaben aus dem Alphabet $\Sigma=\{a,b,c,d,e\}$ mit den Wahrscheinlichkeiten $P(a)=0,15$, $P(b)=0,04$, $P(c)=0,26$, $P(d)=0,05$ und $P(e)=0,5$.
\begin{enumerate}
\item Berechnen Sie die Entropie der Quelle!
\item Bestimmen Sie einen Huffmann-Code f"ur diese Quelle per Hand!
\end{enumerate}

\bigskip

{\bf Aufgabe 2  \quad(Programmieraufgaben)}\\
Bitte beachten Sie die Hinweise zu den Programmieraufgaben.

\smallskip
Schreiben Sie Routinen, die folgende Aufgaben erf"ullen:
\begin{enumerate}
\item Einfache Statistik: Gibt an, welche Bytes in einer Datei vorkommen und wie oft jedes Byte vorkommt.
\item Entropie: Eingabe eine Liste von Zeichenh"aufigkeiten einer Quelle, Ausgabe die Entropie der Quelle.
\item Entropie: Berechnet die Entropie einer Datei, wobei als Zeichen Bytes verwendet werden.
\item(*) Erweitern Sie obige Routinen, so dass sie auch mit Alphabeten von $n$ Bytes arbeiteten. Hat die Eingabe nicht eine L"ange, die ein vielfaches von $n$ ist, so padden sie sie mit 0-Bytes.
\end{enumerate}

\bigskip

{\bf Aufgabe 3  \quad(Empirische Tests)}\\
\begin{enumerate}
\item(*) Generieren Sie eine Datei, bei der der Kompressionsquotient gr"o{\ss}er als 1 ist!
\item\label{a} Laden sie die Beispieldateien zum "Ubungsblatt herunter und nutzen Sie Ihr Programm um die Entropien der Dateien byteweise zu ermitteln!
\item Berechnen Sie mit Hilfe der Entropie die erwarteten minimalen Codel"angen der \verb$*.tex$-Dateien. Wenden sie ein Programm wie \verb$gzip$ auf diese Dateien an und Vergleichen sie deren Gr"o{\ss}e mit der erwarteten minimalen Codel"ange.

Interpretieren Sie das Ergebnis!
\item(*) Ermitteln Sie die Entropien der Dateien aus Teil \ref{a} f"ur Alphabetgr"o{\ss}en \`a 3 und 5 Byte!
\end{enumerate}

\bigskip

{\bf Aufgabe 4  \quad(Arithmetisches Kodieren)}\\
In dieser Aufgabe greifen wir dem Stoff ein wenig vor. Ihre Aufgabe ist es,
Funktionen zum arithmetschen Kodieren eines Strings zu entwerfen.

\bigskip

Bei der arithmetischen Kodierung wird einer Folge von Zeichen eine Dezimalzahl
im Intervall $[0,1)$ bzw. ein Bruch zugeordnet. Dazu wird jedem Zeichen Anhand
der H"aufigkeit ein Bereich in dem Intervall zugeordnet, dann wird beim Einlesen
eines Zeichens das Startintervall auf den Bereich zusammengeschrumpft. Das
neue Intervall wird nun wieder in Bereiche unterteilt, und der Algorithmus wird
rekursiv angewandt. Ist das letzte Zeichen eingelesen, wird eine beliebige
Zahl im "ubriggebliebenen Intervall als Ausgabe gew"ahlt, zusammen mit der
Anzahl der Zeichen die eingelesen wurden.

Decodierung erfolgt analog: Mit der Ausgabe und der Anzahl der Zeichen wird
"uberpr"uft in welchem Teilintervall von $[0,1)$ die Zahl liegt. Dann wird
das Zeichen dieses Teilintervalls an die Ausgabe geh"angt, das Teilintervall
wird nun betrachtet, wieder in Bereiche aufgeteilt und der Algorithmus wird
rekursiv angewendet, bis die Anzahl der gelesenen Zeichen der L"ange der
Ausgabe entspricht.

\bigskip

Beispiel: Zeichensatz ${a, b}$, Eingabe zu Codieren: 'abba'. Das Intervall
wird aufgeteilt in zwei H"alften, die untere H"alfte f"ur $a$, die obere
f"ur $b$. Wir fangen an mit Intervall $[0,1)$, lesen Zeichen 'a'. Das neue
Intervall wird daher $[0,0.5)$. Das n"achste Zeichen ist 'b', daher wird
das neue Intervall $[0.25,0.5)$. Wieder 'b', f"uhrt zu $[0.375,0.5)$.
Das letzte Zeichen, 'a' f"uhrt zum Intervall $[0.375,0.4375)$. Aus diesem
Intervall w"ahlen wir jetzt eine beliebige Zahl als R"uckgabewert, z.B.
$0.375$, die L"ange der Codierten Zeichen ist damit $4$.

R"uckrichtung: gleicher Zeichensatz und Verteilung, Zahl ist $0.375$, L"ange
ist $4$. Wir fangen an mit dem Intervall $[0,1)$. $0.375$ ist in der
unteren H"alfte, d.h. zur Ausgabe kommt 'a' hinzu und das neue Intervall ist
$[0,0.5)$. $0.375$ ist hier in der oberen H"alfte, d.h. ein 'b' kommt zur
Ausgabe und wir erhalten $[0.25,0.5)$. Wir wenden das weiter an, bis wir
die L"ange der Ausgabe erhalten haben, da sind wir wieder im Intervall
$[0.375,0.4375)$, die Ausgabe ist 'abba'.

\bigskip

F"ur Ihre Implementation d"urfen Sie den Zeichensatz einschr"anken, soweit
Sie dies dokumentieren. Alle lateinischen Buchstaben in Standard-ASCII
sollten allerdings vertreten sein. Verwenden Sie eine gleichm"assige
Verteilung der Zeichen auf das Intervall zur Vereinfachung der Implementation.

Verwenden Sie f"ur die R"uckgabewerte entsprechend Ihrer Programmiersprache
typische Konstrukte: In Java bietet es sich an, f"ur den R"uckgabewert eine
Klasse zu definieren. In C oder Pascal w"ahren Structs bzw. Records sinnvoll,
in LISP-Dialekten stattdessen Listen.

\begin{enumerate}
\item (*) Schreiben Sie Routinen zur Codierung des Strings: Gegeben ein String
      \verb$s$ soll die Funktion \verb$EncodeArithmetic$ entweder eine
      Dezimalzahl im Intervall $[0,1)$ oder einen Bruch zur"uckgeben, der wie
      beschrieben codiert ist. Weiterhin soll die Funktion die L"ange der
      Eingabe zur"uckgeben.
\item (*) Schreiben Sie Routinen zur Decodierung: Gegeben dem R"uckgabewert der
      \verb$EncodeArithmetic$ Funktion soll die \verb$DecodeArithmetic$ Funktion
      den String rekonstruieren.
\item (*) "Uberpr"ufen Sie Ihre Implementation mit kurzen und langen Strings.
\item (*) "Uberlegen Sie sich, welche Probleme bei der Implementierung im
      Computer auftreten k"onnen.
\item (*) "Uberlegen Sie sich, wie Sie das Verfahren von dieser vereinfachten
      arithmetischen Codierung verbessern k"onnen. "Uberlegen Sie sich dabei
      getrennt Verbesserungen in Ausgabegr"osse und
      Codierungs-/Decodierungsaufwand.
\end{enumerate}

\bigskip

{\bf Hinweise zu den Programmieraufgaben}\\
Die f"ur die Programmieraufgaben vorgesehene Programmiersprache ist Java. Wenn Sie die "Ubungen in einer anderen Programmiersprache abgeben wollen, so wenden Sie sich an den Betreuer der "Ubungsgruppen.

Allgemein: Die Abgaben erfolgen elektronisch in Form von gut dokumentiertem und compilierf"ahigen Quellcode.

Sofern nicht anders angegeben, sollen die geforderten Programme die zu bearbeitende Datei als ersten Kommandozeilenparameter erhalten.

Die auf diesem Blatt erstellten Routinen werden in den weiteren "Ubungen gebraucht werden. Erstellen Sie die in Aufgabe 2 geforderten Aufgaben in Biblotheksform und erstellen Sie mit deren Hilfe die f"ur Aufgabe 3 geforderten Programme.

Nutzen Sie f"ur das Einlesen der Datei in Java die Klasse \verb$FileInputStream$, um die Datei in Form von Bytes zu einzulesen.

Unter Windows k"onnen die Ergebnisse wegen der unterschiedlichen Darstellung des Zeilenendsymbols abweichen.
\end{document}

