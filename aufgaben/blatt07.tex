\documentclass[a4paper]{article}
\usepackage[top=3.5cm,right=2.75cm,left=2.75cm,bottom=3.5cm]{geometry}

\usepackage[german]{babel}
%\usepackage{umlaut}
\usepackage{graphicx}
\usepackage{amsfonts}
%\usepackage{fullpage}
\usepackage[latin1]{inputenc}

\def\header#1#2#3#4{\pagestyle{empty}
\noindent
\begin{minipage}[t]{0.6\textwidth}
\begin{flushleft}
\bf \"Ubungen zur Datenkompression\\
WSI f\"ur Informatik\\
Krebs/Deininger
\end{flushleft}
\end{minipage}
\begin{minipage}[t]{0.4\textwidth}
\begin{flushright}
\bf Sommersemester 2011\\
Universit\"at T\"ubingen\\
#2 %Datum eintragen
\end{flushright}
\end{minipage}

\begin{center}
{\Large\bf Blatt #1}

{(Abgabe am #3)}
\end{center}
}

\begin{document}
\header{7}{30.06.2011}{10.07.2011}{}

\bigskip

{\bf Zum Blatt}\\
Ziel dieses Blattes ist die Entwicklung eines Vektorquantisierers, der f"ur RGB
Bilder verwendet werden kann.

Als Beispielbild f"ur die Aufgaben dient uns das Foto auf der Startseite des
Lehrstuhls:\\
http://fuseki.informatik.uni-tuebingen.de/images/stories/gruppe-ws0809-small.jpg

\bigskip

{\bf Aufgabe 1  \quad(Vektorquantisierung)}\\
Schreiben Sie Klassen, welche ein Bild als RGB-Farbbild einlesen und
damit ein W"orterbuch trainieren k"onnen:
Betrachten Sie die RGB Farbwerte der Pixel als $(R,G,B)$-Vektoren, und
implementieren Sie die Lindo-Buzo-Gray- und Equitz-Verfahren, welche auf den
Folien zur Vorlesung beschrieben sind.

Halten Sie die Gr"osse des W"orterbuchs konfigurierbar.

\bigskip

{\bf Aufgabe 2  \quad(Vektorquantisierung)}\\
Schreiben Sie ein Programm, welches das in Aufgabe 1 trainierte W"orterbuch
verwendet und auf ein Bild anwendet. Das heisst, jeder Pixel wird auf einen
Eintrag im W"orterbuch 'gerundet' und es wird nur der Index im W"orterbuch
gespeichert. Legen Sie W"orterbuch und Indexfeld wieder mit den
Serialisierungsfunktionen Ihrer Sprache in einer Datei ab, und schreiben Sie ein
weiters Programm welches die Umkehrung berechnet.

\bigskip

{\bf Aufgabe 3  \quad(Beobachtungen)}\\
Vergleichen Sie die Bildausgaben von Aufgabe 2 bei Verwendung verschiedener
W"orterbuchgr"ossen (16, 32, 64 Eintr"age) und die beiden Verfahren
untereinander. Vergleichen Sie weiter die Bildausgaben bei Verwendung von
'falschen' W"orterb"uchern, also zB indem Sie das W"orterbuch mit einem anderen
Bild als dem zu kodierenden trainieren, oder indem Sie ein zuf"alliges
W"orterbuch generieren.

Diskutieren Sie Ihre Ergebnisse.

\end{document}

