\documentclass[a4paper]{article}
\usepackage[top=3.5cm,right=2.75cm,left=2.75cm,bottom=3.5cm]{geometry}

\usepackage[german]{babel}
%\usepackage{umlaut}
\usepackage{graphicx}
\usepackage{amsfonts}
%\usepackage{fullpage}
\usepackage[latin1]{inputenc}

\def\header#1#2#3#4{\pagestyle{empty}
\noindent
\begin{minipage}[t]{0.6\textwidth}
\begin{flushleft}
\bf \"Ubungen zur Datenkompression\\
WSI f\"ur Informatik\\
Krebs/Deininger
\end{flushleft}
\end{minipage}
\begin{minipage}[t]{0.4\textwidth}
\begin{flushright}
\bf Sommersemester 2011\\
Universit\"at T\"ubingen\\
#2 %Datum eintragen
\end{flushright}
\end{minipage}

\begin{center}
{\Large\bf Blatt #1}

{(Abgabe am #3)}
\end{center}
}

\begin{document}
\header{5}{26.05.2011}{05.06.2011}{}

\bigskip
{\bf Zum Blatt}\\
Sie d"urfen sich wieder wie auf Blatt 4 vier der Aufgaben aussuchen. Beachten
Sie, dass Sie f"ur dieses Blatt bis Sonntag um Mitternacht Zeit haben.

\bigskip

{\bf Aufgabe 1  \quad(Entropie)}\\
Zeigen Sie: Ist $X$ eine echt unabh"angige Zufallsquelle ohne Ged"achtnis (die
Wahrscheinlichkeit f"ur ein Zeichen ist unabh"angig von den vorangegangenen
Zeichen), so gilt:

$$2\cdot H(X)= H(X^2).$$
Dabei ist $X^2$, die Zufallsquelle, welche entsteht, wenn man statt einem
Zeichen immer zwei aufeinanderfolgende Zeichen betrachtet.

\bigskip

{\bf Aufgabe 2  \quad(Verh"altnis Signal zu Verzerrung)}\\
Schreiben Sie Routinen, die aus zwei Folgen von {\rm double}-Werten folgende
Werte berechnen: Mittlerer quadratischer Fehler, Signal to Noise und Signal to
Noise in Dezibel.

\bigskip

{\bf Aufgabe 3  \quad(Huffman)}\\
Analog zu Ihrer Shannon-Fano-Kodierung sollen Sie hier die Huffman Kodierung
implementieren.
\begin{enumerate}
\item Warum ist Huffman Kodierung (nicht) zur online Komprimierung geeignet?
      Also z.B. zur Komprimierung von interaktiven Netzwerkverbindungen wie SSH.
\item Implementieren und testen Sie Kodierungs- und Dekodierungsfunktionen zur
      Huffman Kodierung. Als Ausgabe f"ur den Kodierer d"urfen Sie Strings von
      0 und 1 statt Bitfolgen verwenden.
\item(*) Implementieren Sie Ihre Huffman Kodierung so, dass Sie Bitfolgen statt
      0 und 1 Strings ausgibt.
\end{enumerate}

\bigskip

{\bf Aufgabe 4  \quad(BZIP2)}\\
Betrachten Sie das Kompressionsverfahren bei BZIP2: Auf eine Eingabe wird
zun"achst eine Burrows-Wheeler-Transformation angewandt, danach die
Move-to-Front-Transformation und anschliessend die Huffman Kodierung.

\begin{enumerate}
\item Warum ist BZIP2 (nicht) zur online Komprimierung geeignet? Also z.B. zur
      Komprimierung von interaktiven Netzwerkverbindungen wie SSH.
\item Implementieren Sie BZIP2: Basierend auf Ihren L"osungen der letzten
      Bl"atter, wenden Sie auf eine Eingabefolge nacheinander die BWT, die MtF
      und schliesslich die Huffman Kodierung an.
\item Implementieren Sie die Umkehrung von (ihrem) BZIP2.
\item Erweitern Sie Ihren Kodierer um Speicher- und Ladefunktionen.
      Verwenden Sie dazu wieder die Serialisierungsfunktionen Ihrer Sprache.
\item(*) Stellen Sie Ihr BZIP2 auf die Probe: Wenden Sie Ihr Programm in einer
      Schleife auf die Beispieldaten vom ersten Blatt an. Wenden Sie ebenfalls
      das ,,echte'' BZIP2-Programm auf die Daten an. Vergleichen Sie die
      Gr"ossen der Eingabe und der Ausgaben gegeneinander. Diskutieren Sie das
      Ergebnis.
\end{enumerate}

Hinweis: Sollten Sie Aufgabe 3 nicht bearbeitet haben, so nehmen Sie statt dem
Huffman Kodierer aus dieser Aufgabe Ihren Shannon-Fano Kodierer.

\bigskip

{\bf Aufgabe 5  \quad(BZIP2)}\\
Betrachten Sie das Kompressionsverfahren bei BZIP2 wie bei Aufgabe 4.

Wir erweitern das Verfahren um einen Gl"attungsschritt zwischen der BWT und
der MtF. Das heisst, nach der BWT wird der String durchlaufen und jedes einzeln
vorkommende Zeichen wird durch seinen Vorg"anger ersetzt. So wird aus
,,aabaaabbbaacccacaa'' die Zeichenfolge ,,aaaaaabbbaacccccaa''. Danach l"auft
der BZIP2 ab wie beschrieben. Durch diese "Anderung wird ein h"oherer
Kompressionsgrad bewirkt, allerdings wird beim Dekodieren nicht mehr exakt das
Original errechnet sondern nur eine Ann"aherung.

\begin{enumerate}
\item Implementieren Sie diese Variante des BZIP2.
\item Errechnen Sie den Fehler nach Anwendung des modifizierten BZIP2 und seiner
      Umkehrung. Verwenden Sie die Beispieldaten von Blatt 1 und diskutieren Sie
      die Ergebnisse.
\item Vergleichen Sie die Kompressionsgrade des normalen BZIP2 und des
      modifizierten BZIP2. Diskutieren Sie die Ergebnisse.
\end{enumerate}

\bigskip

{\bf Aufgabe 6  \quad(Fehlerbetrachtung)}\\
Betrachten Sie den modifizierten BZIP2 aus Aufgabe 5.

Angenommen es wird genau 1 Zeichen im Gl"attungsschritt ver"andert. Berechnen
Sie, wie viele Zeichen sich nach der Dekodierung vom Original unterscheiden
k"onnen.

Angenommen, Sie h"atten den normalen BZIP2 ohne Gl"attung, und nach Anwendung
des Huffmancodes w"urde ein Bit ver"andert. Wie viele Zeichen k"onnen sich
nun im Original ver"andert haben?

Vergleichen und diskutieren Sie Ihre Ergebnisse.

Hinweis: Sie m"ussen hierf"ur Aufgabe 5 nicht gel"ost haben, es ist eine rein
theoretische Abgabe gefragt.

\end{document}

