\documentclass[a4paper]{article}
\usepackage[top=3.5cm,right=2.75cm,left=2.75cm,bottom=3.5cm]{geometry}

\usepackage[german]{babel}
%\usepackage{umlaut}
\usepackage{graphicx}
\usepackage{amsfonts}
%\usepackage{fullpage}
\usepackage[latin1]{inputenc}

\def\header#1#2#3#4{\pagestyle{empty}
\noindent
\begin{minipage}[t]{0.6\textwidth}
\begin{flushleft}
\bf \"Ubungen zur Datenkompression\\
WSI f\"ur Informatik\\
Krebs/Deininger
\end{flushleft}
\end{minipage}
\begin{minipage}[t]{0.4\textwidth}
\begin{flushright}
\bf Sommersemester 2011\\
Universit\"at T\"ubingen\\
#2 %Datum eintragen
\end{flushright}
\end{minipage}

\begin{center}
{\Large\bf Blatt #1}

{(Abgabe am #3)}
\end{center}
}

\begin{document}
\header{3}{12.05.2011}{19.5.2011}{}

\bigskip

{\bf Aufgabe 1  \quad(RLE) }

Die {\em Laufl"angenkodierung}, englisch {\em run-length encoding} ({\em RLE}),
wird oft zusammen mit anderen Verfahren zur Komprimierung eingesetzt.

In dieser Aufgabe erstellen Sie eine Klasse {\em RLE}, welche genau diese
Kodierung implementiert.

\begin{enumerate}
\item Schreiben Sie eine Routine {\em EncodeRLE}, welche einen String "ubergeben
      kriegt und eine Liste (oder ein Array) von Tupeln der Form $(Z,n)$
      zur"uckgibt, wobei $Z$ das eingelesene Zeichen ist und $n$ die Anzahl, wie
      oft das Zeichen wiederholt wird.
\item Schreiben Sie die Umkehrfunktion {\em DecodeRLE}, welche die Ausgabe von
      {\em EncodeRLE} "ubergeben kriegt und daraus den Originalstring errechnet.
\item Testen Sie Ihre Implementation.
\item Welche anderen Kodierungen kennen Sie, mit denen Sie den Wirkungsgrad der
      Laufl"angenkodierung verbessern k"onnten? Begr"unden Sie.
\end{enumerate}

\bigskip

{\bf Aufgabe 2  \quad(RLE)}

In dieser Aufgabe erweitern Sie Ihre {\em RLE}-Klasse aus Aufgabe 1.

\begin{enumerate}
\item Erweitern Sie Ihre {\em RLE}-Klasse, sowie den R"uckgabewert der
      {\em EncodeRLE}-Funktion so, dass Sie beide jeweils in einer Datei ablegen
      und wieder auslesen k"onnen. Es ist dabei ausdr"ucklich gestattet, dass
      Sie daf"ur Features ihrer Sprache verwenden, z.B. die Serialisierungs-API
      in Java.
\item Erweitern Sie die Klasse so, dass Sie ein beliebiges Alphabet f"ur die
      Kodierung und Dekodierung vorgeben k"onnen. Das Alphabet soll auch
      kurze Zeichenfolgen als einzelne Zeichen im Sinne der Kodierung erlauben.
\item(*) Erweitern Sie die Klasse so, dass Sie Bit-f"ur-Bit arbeiten k"onnen.
\end{enumerate}

\bigskip

{\bf Aufgabe 3  \quad(PPM)}

Es ist Ihnen bei dieser Aufgabe freigestellt, ob Sie die Abgabe von Hand oder
mit einem Computerprogramm l"osen wollen. Die Abgabe erfordert daher keinen
Programmcode.

\begin{enumerate}
\item Berechnen Sie die Wahrscheinlichkeitsmodelle nullter und erster Ordnung
      f"ur die unten stehenden Beispiels"atze. Verwenden Sie hierf"ur geeignete
      Tabellen. Es gen"ugt, wenn Sie sich zwei der S"atze herauspicken.
\item Wie Sie bemerkt haben, handelt es sich bei den Beispiels"atzen um
      Pangramme. Welche Vorteile ergeben sich dadurch f"ur eine arithmetische
      Kodierung die auf Wahrscheinlichkeitsmodellen von diesen S"atzen beruhen?
\item Kodieren Sie zwei beliebige W"orter (mindestens f"unf Buchstaben)
      arithmetisch unter Benutzung der Wahrscheinlichkeitsmodelle die Sie
      berechnet haben, d.h. berechnen Sie jedes Wort erst mit den Modellen von
      einem Satz, dann mit denen von einem anderen.
\item Welche Unterschiede zwischen den Kodierungen stellen Sie fest? Erkl"aren
      Sie.
\end{enumerate}

Beispiels"atze:
\begin{enumerate}
\item Amazingly few discotheques provide jukeboxes.
\item Bored? Craving a pub quiz fix? Why, just come to the Royal Oak!
\item The quick onyx goblin jumps over the lazy dwarf.
\item The wizard quickly jinxed the gnomes before they vaporized.
\end{enumerate}

\end{document}
